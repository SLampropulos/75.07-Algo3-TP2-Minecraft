\documentclass[titlepage,a4paper]{article}

\usepackage{a4wide}
\usepackage[colorlinks=true,linkcolor=black,urlcolor=blue,bookmarksopen=true]{hyperref}
\usepackage{bookmark}
\usepackage{fancyhdr}
\usepackage[spanish]{babel}
\usepackage[utf8]{inputenc}
\usepackage[T1]{fontenc}
\usepackage{graphicx}
\usepackage{float}

\pagestyle{fancy}
\fancyhf{}
\fancyhead[L]{TP2 - AlgoCraft}
\fancyhead[R]{Algoritmos y Programación III - FIUBA}
\renewcommand{\headrulewidth}{0.4pt}
\fancyfoot[C]{\thepage}
\renewcommand{\footrulewidth}{0.4pt}

\begin{document}
\begin{titlepage}
	\hfill\includegraphics[width=6cm]{logofiuba.jpg}
    \centering
    \vfill
    \Huge \textbf{Trabajo Práctico 2 — AlgoCraft}
    \vskip2cm
    \Large [7507/9502] Algoritmos y Programación III\\
    Curso 1 \\
    Primer cuatrimestre de 2019 
    \vfill
   \begin{tabular}{ |l|l|l| }
		\hline
		\multicolumn{3} { |c| } {\textbf{Integrantes del grupo}} \\ \hline
		 DAVEREDE Agustín & 98540 & agusdavi64@gmail.com\\ \hline
	 	HUENUL Matías & 102135 & matias.huenul.07@gmail.com\\ \hline
		HUZAN Hugo & 67910 & hhuzan@gmail.com\\ \hline
		LAMPROPULOS Santiago & 101862 & santiagolampropulos@gmail.com\\ \hline
\end{tabular}
    \vfill
    \vfill
\end{titlepage}

\tableofcontents
\newpage

\section{Introducción}\label{sec:intro}
En el presente informe se expone la aplicación desarrollada para el segundo trabajo práctico de la materia, en lenguaje Java, y los conceptos teóricos de la programación orientada a objetos utilizados en el diseño de la misma.


\section{Supuestos}\label{sec:supuestos}
Debido a detalles no especificados en la consigna del trabajo práctico, se ha decidido adoptar los siguientes supuestos.
\begin{itemize}
\item Supuesto 1...
\item Supuesto 2...
\end{itemize}


\section{Modelo de dominio}\label{sec:modelo}
El diseño del trabajo consiste principalmente en las siguientes clases.
\begin{description}
\item[Jugador] Modela al jugador de la partida, que puede moverse en el mapa del juego. Posee un inventario de herramientas y materiales. Puede recolectar materiales y luego usarlos para fabricar herramientas.
\item[Mapa] Modela al mundo en el cual el jugador puede moverse. El mapa es un conjunto de celdas, las cuales pueden estar vacías o ocupadas, ya sea por el jugador o por distintos materiales.
\item[Herramienta] Es una clase abstracta que modela una herramienta genérica. Posee una durabilidad y una fuerza determinadas por el tipo específico de herramienta. Puede ser usada en materiales, reduciendo la durabilidad de éstos y también la propia.
\item[Material] Es una clase abstracta que modela un material. Posee una durabilidad, que puede ser desgastada por una herramienta. Se encuentran distribuidas en el mapa del juego y al reducirse por completo su durabilidad puede ser obtenida por el jugador.
\end{description}

\section{Diagramas de clase}\label{sec:diagramasdeclase}
A continuación se encuentran los diagramas que muestran las clases implementadas y cómo se relacionan entre sí.

\begin{figure}[H]
\centering
\includegraphics[width=\textwidth]{Diagramas/Materiales.png}
\caption{\label{fig:material}Diagrama de clases de Material.}
\end{figure}

\begin{figure}[H]
\centering
\includegraphics[width=\textwidth]{Diagramas/Herramienta.png}
\caption{\label{fig:herramienta}Diagrama de clases de Herramienta.}
\end{figure}

\begin{figure}[H]
\centering
\includegraphics[width=\textwidth]{Diagramas/Desgastador.png}
\caption{\label{fig:desgastador}Diagrama de clases de Desgastador.}
\end{figure}

\begin{figure}[H]
\centering
\includegraphics[width=\textwidth]{Diagramas/Jugador.png}
\caption{\label{fig:jugador}Diagrama de clases de Jugador.}
\end{figure}

\begin{figure}[H]
\centering
\includegraphics[width=\textwidth]{Diagramas/Mapa.png}
\caption{\label{fig:mapa}Diagrama de clases de Mapa.}
\end{figure}

\section{Detalles de implementación}\label{sec:implementacion}

\section{Excepciones}\label{sec:excepciones}

\section{Diagramas de secuencia}\label{sec:diagramasdesecuencia}

\end{document}